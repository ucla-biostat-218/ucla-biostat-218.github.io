% OHDSI slides template
\documentclass[11pt
%,serif,marthserif
,fragile=singleslide
%%dvips
%%,draft
,xcolor=dvipsnames, 
aspectratio=169 % update this if you want narrow screen; 16:9 wide screen is recommended
]{beamer}

%% include the preamble and all setup
%% preamble setup for OHDSI beamer template
\usepackage[]{graphicx}
%% maxwidth is the original width if it is less than linewidth
%% otherwise use linewidth (to make sure the graphics do not exceed the margin)
\makeatletter
\def\maxwidth{ %
	\ifdim\Gin@nat@width>\linewidth
	\linewidth
	\else
	\Gin@nat@width
	\fi
}
\makeatother

\definecolor{fgcolor}{rgb}{0.345, 0.345, 0.345}
\newcommand{\hlnum}[1]{\textcolor[rgb]{0.686,0.059,0.569}{#1}}%
\newcommand{\hlstr}[1]{\textcolor[rgb]{0.192,0.494,0.8}{#1}}%
\newcommand{\hlcom}[1]{\textcolor[rgb]{0.678,0.584,0.686}{\textit{#1}}}%
\newcommand{\hlopt}[1]{\textcolor[rgb]{0,0,0}{#1}}%
\newcommand{\hlstd}[1]{\textcolor[rgb]{0.345,0.345,0.345}{#1}}%
\newcommand{\hlkwa}[1]{\textcolor[rgb]{0.161,0.373,0.58}{\textbf{#1}}}%
\newcommand{\hlkwb}[1]{\textcolor[rgb]{0.69,0.353,0.396}{#1}}%
\newcommand{\hlkwc}[1]{\textcolor[rgb]{0.333,0.667,0.333}{#1}}%
\newcommand{\hlkwd}[1]{\textcolor[rgb]{0.737,0.353,0.396}{\textbf{#1}}}%

\usepackage{framed}
\makeatletter
\newenvironment{kframe}{%
	\def\at@end@of@kframe{}%
	\ifinner\ifhmode%
	\def\at@end@of@kframe{\end{minipage}}%
\begin{minipage}{\columnwidth}%
	\fi\fi%
	\def\FrameCommand##1{\hskip\@totalleftmargin \hskip-\fboxsep
		\colorbox{shadecolor}{##1}\hskip-\fboxsep
		% There is no \\@totalrightmargin, so:
		\hskip-\linewidth \hskip-\@totalleftmargin \hskip\columnwidth}%
	\MakeFramed {\advance\hsize-\width
		\@totalleftmargin\z@ \linewidth\hsize
		\@setminipage}}%
{\par\unskip\endMakeFramed%
	\at@end@of@kframe}
\makeatother

\definecolor{shadecolor}{rgb}{.97, .97, .97}
\definecolor{messagecolor}{rgb}{0, 0, 0}
\definecolor{warningcolor}{rgb}{1, 0, 1}
\definecolor{errorcolor}{rgb}{1, 0, 0}
\newenvironment{knitrout}{}{} % an empty environment to be redefined in TeX

\usepackage{alltt}
\hypersetup{pdfpagemode=FullScreen}

% \mode<presentation>
% {
%   \usetheme{Warsaw}
%   \setbeamercovered{transparent}
%   % or whatever (possibly just delete it)
% }

\usepackage[english]{babel}
% or whatever

\usepackage[utf8]{inputenc}
% or whatever

\usepackage{listings}
\usepackage{times}
\usepackage{color}
\usepackage{wasysym}
\usepackage{amstext, amssymb, amsthm, amsmath,graphicx,rotating,array}
\usepackage{algorithmic,algorithm,booktabs,url,colortbl}
%\usepackage{natbib} % commented out due to weird errors
\usepackage[T1]{fontenc}
\usepackage{amsmath}

\usepackage{eso-pic}
% \usepackage[absolute,overlay]{textpos}
\usepackage{textpos}

\usepackage{tikz}
\usepackage{pgf}
%\usepackage{cancel}
\usepackage{soul}

%\usepackage{movie15}
% Or whatever. Note that the encoding and the font should match. If T1
% does not look nice, try deleting the line with the fontenc.

\newcommand{\argmax}{\operatornamewithlimits{argmax}}

\definecolorset{rgb}{}{}{darkred,0.8,0,0;darkgreen,0,0.5,0;darkblue,0,0,0.5}
\newcommand{\bm}[1]{\mathbf{#1}}

\usepackage{color}
\usepackage{ulem}
\usepackage{wrapfig}
\usepackage{algorithm}
\usepackage{algorithmic}
\usepackage{inputenc}

% START

%\definecolor{Red}{rgb}{1,0,0}
\definecolor{ohdsiblue}{rgb}{0.125,0.259,0.353} %{0.0,0.204,0.416}
\definecolor{darkgreen}{rgb}{0.1,0.6,0.1}

\definecolorset{rgb}{}{}{darkred,0.8,0,0;darkgreen,0,0.5,0;darkblue,0,0,0.5}

\definecolor{trevorblue}{rgb}{0.330, 0.484, 0.828}
\definecolor{trevoryellow}{rgb}{0.829, 0.680, 0.306}
\definecolor{yaleblue}{rgb}{0.0,0.204,0.416}

\usetheme{Boadilla}
\usecolortheme{beaver}
\usenavigationsymbolstemplate{} % no navigation symbols

\setbeamertemplate{enumerate item}[default]

\useoutertheme{infolines}

\newcommand{\myskip}{1em}

% \setbeamercolor{author in head/foot}{fg=ohdsiblue,bg=black!5}
\setbeamercolor{author in head/foot}{fg=ohdsiblue,bg=white}
\setbeamercolor{frametitle}{fg=ohdsiblue,bg=black!5}
\setbeamercolor{block title}{bg=yaleblue!10,fg=black}

\setbeamertemplate{frametitle}{%
	\vskip-1pt
	\begin{beamercolorbox}[sep=0.3cm,wd=\paperwidth]{frametitle}
		\usebeamerfont{frametitle}%
		%\vbox{}\vskip-1ex%
		\vskip-9pt
		\hskip-9pt
		\parbox{1.5cm}{%
			\includegraphics[width=1.5cm]{figures/logo.png}
		}
		%\hskip-2em
		%\hbox{}\vskip-3em
		\strut\insertframetitle\strut\par%
		\vskip-8pt
	\end{beamercolorbox}
}

\def\Put(#1,#2)#3{\leavevmode\makebox(0,0){\put(#1,#2){#3}}}

\setbeamertemplate{footline}[text line]{%
	\hbox{\begin{beamercolorbox}[wd=\paperwidth,ht=1em,dp=1.125ex,leftskip=.3cm plus1fill,rightskip=.3cm]{author in head/foot}%
			\usebeamerfont{author in head/foot}
			{\makebox[\paperwidth]{\hfill\makebox[10pt]{\tiny \insertframenumber}}}
			\centering \includegraphics[width=\paperwidth,height=1em]{figures/footer.png}
			% \centering A \insertauthor \hspace{4em} \insertshortinstitute \hspace{4em} \insertshorttitle
		\end{beamercolorbox}%
} }

%\setbeamertemplate{footline}{%
%	

% \newcommand\AtPageMyUpperLeft[1]{\AtPageLowerLeft{%
% \put(\LenToUnit{0.9\paperwidth},\LenToUnit{0.9\paperheight}){#1}}}
% \AddToShipoutPictureFG{
%   \AtPageMyUpperLeft{{\includegraphics[width=1.5cm,keepaspectratio]{OHDSI_logo_only_colored}}}
% }%

% \usepackage{pgf}
% \logo{\includegraphics[height=1cm]{OHDSI_logo_only_colored}}
% \logo{\pgfputat{\pgfxy(2,2)}{\pgfbox[center,base]{\includegraphics[width=1.7cm]{OHDSI_logo_only_colored}}}}
% \logo{\pgfputat{\pgfxy(0,8)}{\pgfbox[right,base]{\includegraphics[height=1cm]{{OHDSI_logo_only_colored}}}}}
%\logo{\pgfputat{\pgfxy(0,8)}{\pgfbox[right,base]{\includegraphics[height=1cm]{logo.pdf}}}}
% \newcommand{\nologo}{\setbeamertemplate{logo}{}}

%\pgfdeclareimage[height=2.5cm, width=2.5cm]{logo}{OHDSI_logo_only_colored}

% \addtobeamertemplate{frametitle}{}{%
% % \begin{textblock*}{100mm}(.05\textwidth,-1cm)
% % \includegraphics[height=2cm,width=2cm]{OHDSI_logo_only_colored}
% % \end{textblock*}
% \begin{tikzpicture}[remember picture,overlay]
% \node[anchor=north west,yshift=-4pt] at (current page.north west) {\includegraphics[height=1.5cm]{OHDSI_logo_only_colored}};
% \end{tikzpicture}
% }


% \addtobeamertemplate{headline}{}
% {%
% \pgfuseimage{logo}
% % \begin{tikzpicture}[remember picture,overlay]
% % \pgfuseimage{logo}
% % \draw [line width=1mm,opacity=.25]
% % (current page.north east) circle (3cm);
% % \end{tikzpicture}
% }


% \logo{\pgfuseimage{logo}}

% END


% Delete this, if you do not want the table of contents to pop up at
% the beginning of each section:
\AtBeginSection[]
{
  \begin{frame}<beamer>
    \frametitle{Outline}
    \tableofcontents[currentsection,currentsubsection]
  \end{frame}
}


% If you wish to uncover everything in a step-wise fashion, uncomment
% the following command:

%\beamerdefaultoverlayspecification{<+->}


\setcounter{algorithm}{1}
\newcommand{\myalgorithm}[1]{
	\vspace{0.5cm}
	\begin{quote}
		{{\sc Algorithm} \arabic{algorithm}: #1}
	\end{quote}
	\vspace{1cm}
	\stepcounter{algorithm}
}

\newcommand{\HeaderA}[3]{#1 #2 #3}
\IfFileExists{upquote.sty}{\usepackage{upquote}}{}


\usepackage{setspace}

% for highlight
\newcommand{\hlfb}[1]{{\textcolor{Cerulean}{#1}}}
% for blank footnote
\newcommand\blfootnote[1]{%
	\begingroup
	\renewcommand\thefootnote{}\footnote{#1}%
	\addtocounter{footnote}{-1}%
	\endgroup
}


% Specifiy title here
\title[
OHDSI Biostat218
]
{
OHDSI Biostat218
}
\subtitle
{} % (optional subtitle)

\author{Marc A Suchard}

\institute[OHDSI] % (optional, but mostly needed)
{
	Department of Greatness \\
	School of Awesome Health Data Science Research
	% - Use the \inst command only if there are several affiliations.
	% - Keep it simple, no one is interested in your street address.
}
\date{Jan 1, 2024}


% start the slides
\begin{document}
	
	\begin{frame}
		%  \titlepage
		\begin{columns}
			\begin{column}{0.33\textwidth}
				\includegraphics[width=1.0\textwidth]{figures/logo_title_page}
			\end{column}
			\begin{column}{0.66\textwidth}
				\begin{center}
					{\LARGE
						Open Science} \\[2em]
					
						Marc A Suchard, MD, PhD \\[0.5em]
						Department of Biostatistics \\ 
						UCLA Fielding School of Public Health\\[1em]
%						{\small Joint work with other OHDSI collaborators}
					%				}
				\end{center}
			\end{column}
		\end{columns}
		\vspace{1em}
		\vspace{1em}
		
	\end{frame}
	
\begin{frame}
	\frametitle{OHDSI is an open science community}

	\centerline{\includegraphics[width=0.75\textwidth]{figures/evidence_journey}}

	\begin{columns}[T]
	\begin{column}{0.5\textwidth}
	Different observational data types:
	\begin{itemize}
		\item {\small Populations, care settings, capture process, health system}
	\end{itemize}
	\end{column}
	\begin{column}{0.5\textwidth}
	Types of evidence desired:
	\begin{itemize}
		\item {\small Cohort identification, clinical characterization, population-level effects, patient-level prediction}
	\end{itemize}
	\end{column}
	\end{columns}

	\begin{block}{OHDSI's mission}
	\begin{itemize}
\item To improve health, by empowering a community to collaboratively generate the evidence that promotes better health decisions and better care
\end{itemize}
	\end{block}

\end{frame}

\begin{frame}
  \frametitle{Open science movement}

	\begin{block}{Open science aims}
	\begin{itemize}
	\item To make scientific research (including publications, data, samples and software) and its dissemination accessible to all levels of an inquiring society
\end{itemize}
	\end{block}
  
\vspace{1em}  
  
Addresses a number of perceived problems within current scientific practice, namely
\begin{itemize}
  \item Inclusivity of participants
  \item Publication bias, $p$-hacking
  \item Challenges in \textbf{reproducibility} / \textbf{reliability}
\end{itemize}

\vspace{1em}

All important to ``empower a community'' \ldots ``better health decisions''
  
\end{frame}

\begin{frame}
  \frametitle{Four principles of open science}
  
  \begin{itemize}
    \item Open standards \\[2em]
    \item Open source \\[2em]
    \item Open data \\[2em]
    \item Open discourse
  \end{itemize}
\end{frame}

\begin{frame}
  \frametitle{Open standards examples}

  \begin{itemize}
    \item OMOP common data model -- scoped to capture observational healthcare data and now a world-wide standard (PCORnet, AllOfUs \ldots just subsets)  \\[2em]
    \item Standardized vocabularies to map the hundreds of medical coding systems used world-wide   \\[2em]
  \end{itemize}

\begin{block}{Example}
  \centerline{ATHENA -- \url{https://athena.ohdsi.org} (``Ozempic'' $\rightarrow$ ``semaglutide'')}
\end{block}
\end{frame}

\begin{frame}
  \frametitle{Open source examples}
  
  \begin{itemize}
    \item ATLAS to define reproducible patient cohorts \\[2em]
    
    \item Health analytics data-to-evidence suite (HADES) of \texttt{R} packages to perform standardized analyses  \\[2em]
  \end{itemize}
  
\begin{block}{Example}
  \centerline{CohortMethod -- \url{https://github.com/OHDSI/CohortMethod} (read a vignette)}
\end{block}  
    
\end{frame}

\begin{frame}
  \frametitle{Open data examples}
  
  \begin{itemize}
  	\item All population-level summarizes are publicly accessible!  \\[2em]  
  \end{itemize}
  
\begin{block}{Example}
  \centerline{LEGEND-T2DM}
  \centerline{\url{https://data.ohdsi.org/LegendT2dmClassEvidenceExplorer/}}
  \centerline{(DDP4I vs GLP1RA for MACE)}
\end{block}    

\vspace{1em}

NB: Patient-privacy and ethics preclude sharing individual-level information

\end{frame}


\begin{frame}
  \frametitle{Open discourse examples}

  \begin{itemize}
    \item Anyone interested in promoting OHDSI's mission can join \\[2em]
    \item Anyone can lead (former undergraduate workgroup chair) \\[2em]
    \item Public forums, weekly community class, 20+ workgroups \\[2em]
 
\begin{block}{How to join}
  \centerline{\url{https://www.ohdsi.org}}
\end{block}        
    
  \end{itemize}
  
\end{frame}

\end{document}
